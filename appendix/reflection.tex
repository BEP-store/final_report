\chapter{Reflection}
This chapter presents the reflection of the team of on the projects. It is divided into three sections. The first section is about the problems that were presented during the process of solving the problem. Section two describes the particular challenges of finding a solution to the problem. The third section presents the personal reflections by each of the team members.

\section{Process}
During the process of implementing a solution to the problem, the biggest problem proved to be planning. Although planning and process management have been a point of attention from the start of the project, the biggest challenge turned out to be correctly assessing the amount of time needed to implement features. During the weekly scrum meetings, it often turned out that some features were more complex than previously assessed, resulting in delays during that iteration that were carried over to the next iteration.

A simple solution to the problem described above might be to simple move on from delayed features and deal with the next iteration as a separate concern. However, in software development this is not always possible since future functionalities might depends on features from the current iteration. Another solution might be to revise the planning once a delay occurs. This solution was applied by the team, and although the implemented solution did not adhere to every requirement, the result did satisfy the most important requirements.

\section{Solution}
The solution found as a result of the research can be applied in other contexts than that of the client. In addition, the research presented findings that the team found personally interesting. This resulted in the team being more engaged in the subject and the project. This was emphasised by the educational context of the problem and its relation to software development. With regards to the implemented solution, the team would have like to do more. Although it can be expected that a planning might change, the team became invested in the project, often placing personal plans second to put the project first. 

\section{Challenges}
In implementing the solution, the team tried to follow the philosophy and architecture behind the client's platform. In doing so, they were presented with a few challenges. For some of the team members, understanding the architecture of the client's platform proved difficult at first due to the fragmented nature of the platform. To other team members, a lack of experience in commercial software development was a barrier to overcome. However, due to the support of other team members and the general collaborative effort between team members, these problems were overcome over the course of the project.

\section{Personal reflection}
The following sections describe the personal reflection of the team members of this project. It is written from a first person perspective. The contents of the reflection represent several aspects of the project, ranging from prior knowledge to satisfaction with the implemented solution.

\subsection{Bart Heemskerk}
At the start of the project I had no experience with either Ember or Ruby, so I had a lot of catching up to do in order to understand the platform of the client, let alone build an application using the styles already provided by the client. Part of the application was planned to be integrated in the platform of the client, which was a basis as well as a restriction. When the actual coding started, I started creating the basis of the user interface en practically never left.

Designing and implementing the user interface was a new experience because I used to create the code which gave the user interface its functionality without bothering much how the application worked. Having the styles from the client helped me create a basis, after which we could create our solution as we intended to. Thanks to the rest of the team I was quickly able to implement every component in a rapid pace. I am thankful for the patient for the patient they had when I had another question.

Although the solution lacks some of the intended functionalities, I think that the solution can fulfil its intended purpose. The solution enables the users of the platform of the client to address the lack of a certain functionality, feedback that is valuable for the client. The solution also facilitates that the functionality can be created without the help of the client. Because the solution implements the elements that GitHub lacks and are needed for the development of the micro-services, I think that the solution is an addition to the platform of the client. Therefor I can say that I am proud of what we have created.

\subsection{Wouter Kooyman van Guldener}
Let me start by saying that I enjoyed this project very much. Especially the environment at FeedbackFruits was very inspiring and has helped keeping our motivation high. Also, the fact that we kept everything (even all the reports) open-source stimulates to deliver quality work.

Personally, I think the cooperation with Bart and Steffan went very well, we really complemented each other. Steffan was already familiar with the existing platform and frameworks and could help Bart and me out in several occasions. Bart was really excelling with his design skills. While we all had our own expertise, the use of pull-requests really helped to keep everyone sharp and involved in every aspect of the project.

Improvements to the process could be made though, especially in keeping the planning up to date. It frequently occurred that we were too optimistic in our planning and did not meet the set requirements for a milestone. A solution to this could be to adhere more to the Scrum methodology and assign weights and time indications to each and every issue in the backlog. On the other hand, having a too optimistic planning may not be a bad thing. If you do not get demotivated if you do not meet the planning, it may stimulate someone to work harder to actually meet the target.

Another improvement could be made in the update rounds with Emile Hendriks, our coach. I feel that the the meetings could have been better prepared with respect to showing what we actually did the past period. This does not apply to updating the client because of the Tuesday-evening update rounds at FeedbackFruits.

\subsection{Steffan Sluis}
I had been an employee of the client for over a year at the start of the project. I was closely involved with the inception of the project, which resulted in me acting as a bridge between the client and the team. It also meant the barrier of understanding the subject and the context of the project was substantially lower for me. At the start of the project, this resulted in a lot of collaborative work between me and the rest of team. This changed quickly after starting the implementation phase of the project, since the team picked up on the client's platform architecture very quickly.

The solution that was implemented does not cover all of the intended requirements. In my opinion, the scope of the research and extensibility of the solution more than compensates for this. The target of the project has always been to facilitate the involvement of others in the improvement of software, and the implemented solution reflects the focus on inclusion and extensibility rather than the implementation of particular features. I am happy with the results.
