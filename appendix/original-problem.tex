\chapter{Original project description}
Open source software development has grown during past years. With the adoption of Pull Requests as a means of performing code review and Continuous Integration and Continuous Deployment, an interesting model for software development has arisen: crowd-sourced software development.

On the other hand, micro-service oriented application structures have also become popular over the last years. This application architecture often allows for new functionalities to be added with little effort and with decreased risk of unintended consequences, since each part of the application as a whole functions independently. If one part breaks, it doesn't necessarily affect others.

The project consist of building a platform that streamlines the process of building a micro-service that can interact with the FeedbackFruits API. The goal is to allow developers to immediately create new functionalities, with little knowledge about the internal workings of the API. The platform should offer a set of requirements for functionalities, which can be tested for automatically, as well as the ability to request new functionalities, given a set of requirements.

The platform is targeted at developers who are not (completely) familiar with the codebase of the application they are expanding. For that reason, the main goal of the platform should be to streamline the process from feature request to pull request.

During the project, students may encounter the following problems:
\begin{itemize}[nosep]
    \item What kinds of requirements can be automatically tested for, and what kinds can't?
    \item How and when should communication work between platform, developer, feature requester and application maintainers?
    \item How and when does a service become deprecated? How is it maintained?
    \item What feature requests have priority over others, and why?
    \item How does my service communicate with others?
\end{itemize}

\section{Company description}

We are a startup improving education for learners worldwide. Our platform is built on modern web technologies like Rails, Ember.js, HTML5 MongoDB, Elasticsearch and Git. With more than 30K users and a team of over 20 professional geeks, we work to make education better! We are very passionate and dedicated to our mission. Our devteam ranges from students who hack on education part time to full time, full stack, experienced coders. If you would like to join us in making student life better one pull request at a time and want to dig deeper into web building while doing so, keep on reading!

Our product owner, design and development teams work closely together. We work agile using short 2-week release cycles. A number of our developers have done their own graduation projects within FeedbackFruits, while studying at TU Delft. They are well familiar with the process and will be able to give you valuable advice and guidance. Our 2.0 platform on which you would be working is modularized, meaning that you will have significant freedom in choosing the technology you would like to use in your project. 

\section{Auxiliary information}

You will be able to work on the project from our office located in the startup incubator Yes!Delft. This will allow you to experience the awesome atmosphere in the startup community and being around some of the most innovative businesses in Europe. You will be renumerated for the work on your project and will enjoy all the awesome free lunches, weekly dinners, and other team events like sailing trips and prison breaking.

Our team consists of both dutch and international people, therefore, you will be able to write your proposal in the preferred by you language (Dutch or English). We do not oppose any limitations as to the group size, but would recommend a group size of 2-3 people. One student (R.S. Sluis) has already confirmed his participation in this project.

Last but not least you will have the opportunity to be coached by Emile Hendriks, the director of Computer Science education at TU Delft. Emile has published over 100 papers in international journals and conferences and has supervised over 50 master students and co-supervised 10 PhD students and is one of the leading educational visionaries in the Netehrlands. .