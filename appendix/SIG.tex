\chapter{\texorpdfstring{\Gls{sig}}{SIG} Feedback}
There are two checks performed by \gls{sig}. The feedback for the first check was in Dutch and has been translated. The results for the second check have not yet been received so could not be processed.

\section{Feedback content}
The code of the system scores 4.5 stars on our maintainability-scale, which means that the maintainability of the code is above average. However, improvements can be made when it comes to duplication.

Duplication is determined by looking at the percentage of code which is redundant, primarily the code that has multiple occurrences in the system and can be removed. From the perspective of maintainability, it is desirable to have a low percentage of redundancy, because adjustments to the redundant code have to be done in multiple places.

Multiple duplications are found in the code. For example, the file \textit{click-outside.js} is present two times in the implementation. Duplication was found between the files \textit{show.js}, \textit{edit.js} and \textit{deploy.js}. In the html-code (\textit{422.html}, \textit{404.html} and \textit{500.html}) duplication also occured. The same 45 lines of code are present in all three locations. Try to rewrite the code in a easier way, such that the code can be reused instead of duplicated.

Finally it is good to see that you have written a lot of test-code besides the production-code. The ratio of test-code and production-code is 2:3.

We hope that you are able to maintain the quality of the code as well as the ratio of test-code and production-code during the remainder of the project.''

\subsection{Processing of feedback}
The feedback indicates that the structure of the implementation is good. \Gls{sig} is satisfied with the ratio of test-code and production-code. Combined with the test coverage aggregated by CodeCov, the end-product will be well-tested.

The only negative comments are about code duplications. While the duplication in the html-code were auto-generated by \textit{Ruby on Rails}, they were corrected nonetheless.

The other duplications were in the files \textit{show.js}, \textit{edit.js} and \textit{deploy.js}. However, after investigation it was concluded that \textit{deploy.js} was misspelled and should actually have been \textit{destroy.js}. To remove most of the duplication, the recurring parts were combined into its own function. Part of the duplication is caused by simple import statements. It does not make sense to refactor those.

\subsection{After the feedback}
After the feedback was received, the team was more keen on spotting and avoiding code duplication in (each others) pull requests. Pull requests were only merged if the duplication was removed. According to CodeClimate the \textit{Rails-api} has no duplications at all, and the \textit{Ember-ui} indicates that there are two small duplications. The fixes for these duplications make the code less readable, so it has been decided to not remove those duplications.

% De code van het systeem scoort 4.5 ster op ons onderhoudbaarheidsmodel, wat betekent dat de code bovengemiddeld onderhoudbaar is. Er zijn echter een paar verbeteringspunten bij Duplication.

% Voor Duplicatie wordt er gekeken naar het percentage van de code welke redundant is, oftewel de code die meerdere keren in het systeem voorkomt en in principe verwijderd zou kunnen worden. Vanuit het oogpunt van onderhoudbaarheid is het wenselijk om een laag percentage redundantie te hebben omdat aanpassingen aan deze stukken code doorgaans op meerdere plaatsen moet gebeuren. 

% Er zijn in dit project meerdere duplciaties te vinden: bijvoorbeeld het click-outside.js bestand is twee keer in de upload aanwezig. Verder is duplicatie gevonden tussen de bestanden show.js, edit.js en deploy.js. Ook in de html code (422.html 404.html, 500.html) is de code gekopieerd, dezelfde 45 lines of code zijn aanwezig op alle drie verschillende locaties. Probeer deze code te herschrijven op een slimmere manier, zodat de code wordt hergebruikt in plaats van gedupliceerd. 

% Tot slot is het goed om te zien dat jullie naast productiecode ook redelijk veel testcode hebben geschreven. De verhouding tussen testcode en productiecode is 2:3. 

% Hopelijk lukt het jullie nog om zowel de kwaliteit van de code als de verhouding tussen testcode en productiecode tijdens het vervolg van het project te behouden.