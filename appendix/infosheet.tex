\chapter{Infosheet}
\begin{description}[nosep]
    \item[Title of the project:] The reverse app store
    \item[Client:] L.\ Starovoitova, FeedbackFruits
    \item[Coach:] Dr.\ E.A.\ Hendriks, Opleidingsdirecteur, Computer Science, TU Delft
    \item[Contact person:] W.J.\ Kooyman van Guldener, \url{wouter@wick-it.nl}
    \item[Final presentation date]: June 24th, 2016
\end{description}

\section{Description}
% \subsection{Challenge}
Engagement of non-technical users in the software development process is difficult. Users often have valuable feedback on the clients platform, but the real value of their feedback often gets lost in the translation to the actual software implementation. 

% \subsection{Research}
The requirements for community engagement have been thoroughly researched, with a focus on software communities. Additionally, requirements engineering was also researched to gain insight into feedback aggregation and problem decomposition.

% \subsection{Process}
The team adopted an agile approach for creating an optimal solution in a short period. Many tools were used to ensure a stable production environment and a good code climate. Because the team was based at the client, continuous input of the client made it possible to fully customise the solution to the clients needs.

% \subsection{Product}
The implemented solution is a platform that is closely connected to the clients platform, focused on engaging non-technical users in the process of enhancing the clients platform. With the solution it is possible to create a goal. Through GitHub integration it is possible to create issues and milestones in a readable format for non-technical users. It is also possible to share a goal via Twitter, Facebook and LinkedIn. Communication is provided via Gitter integration. The solution is live\footnote{\url{https://bepstore.feedbackfruits.com}} and will be subjected to a user test in the form of a hackathon.

% \subsection{Outlook}
While the current solution is already live, it can still be improved. A future version of the platform would add more integrations and make it even more easy for users to engage. Focus should be on an improved version of the requirements engineering part and involvement of designers.

\section{Most important contributions per project member}
\begin{description}
    \item[Bart Heemskerk] Responsible for the larger part of the design and usability of the front-end.
    \item[Wouter Kooyman van Guldener] Responsible for the larger part of the back-end and synchronisation between users.
    \item[Steffan Sluis] Responsible for the integration of external services and compatibility with the client's platform.
    \item[All] Responsible for the research report, final report, final project presentation and quality assurance.
\end{description}

“The final report for this project can be found at: 
        "\url{http://repository.tudelft.nl}"


% The Info sheet is a single A4 that contains a description of the project, including information on 
% its unique points, and a short blurb on the role of each of the team members. It should contain the 
% following points: 
% - Title of the project 
% - Name of the client organization 
% - Date of the final presentation 
% - Description 

% Short description of the problem that was tackled by the project, including one or two lines about the client. You may also add a diagram, photo or logos (e.g., TU Delft Logo). Note: if the problem formulation shifted or developed over the course of the project, describe the problem that was actually addressed, and not the problem that was initially intended to be addressed. Please briefly touch on each of these points in your description: 

%     -  Challenge: Statement of the core challenge of the project. 
%     - Research: Statement of what the students learned during the research phase, and how 
%     that informed the decisions in the project. 
%     - Process: Statement of how the process was set up, and any adaptations that needed to be 
%     made while the process was running. Mention (briefly!) any unexpected challenges  and 
%     how they were overcome. 
%     - Product: Description of the product that was created, and how it was tested. 
%     - Outlook: Describe the outlook of the product. Did the team make recommendations to the     client? (if so, briefly summarize) Will the product be used?  
    
    % Members of the project team 
    % For each team member include: 
    % - Name (nB do not include student numbers.) 
    % - One sentence description of the person’s interests (including and going beyond computer science, if relevant) and (optional) of any other relevant facts about experience. (If you really don’t want to list interests, then you can skip this sentence, if you have many, then you can add two sentences.) 
    % - Description of the contributions of the person to the project, and the role played by that person in the project. Be sure to highlight the critical contributions. 
    % If there is something that all team members did, that can be included as a separate sentence at the bottom (in order to avoid redundancy): e.g., “All team members contributed to preparing the report and the final project presentation”. 
    % Please include this information at the bottom of the page: 
    % - Name and affiliation of the Client 
    % - Name and affiliation (Department and Group) of the TU Coach 
    % - Contact person: The email address of one or more people (team members or project 
    % coach) to contact that is expected to be valid at least five years after the completion of the 
    % project. 
    % -  The sentence 
 
% We provide some early examples of Infosheets to convey an idea of what they should contain. 
% Note that these are not necessarily complete or optimal, but rather provide the team with an idea 
% of how to create their own. 
% http://homepage.tudelft.nl/q22t4/Resources/TUDelftCSBachelorProjectInfosheetExamples.pdf 