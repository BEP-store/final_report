% Steffan
\chapter{Client}

The client is a company called FeedbackFruits that offers an online learning platform to help innovate education. The platform is used on a daily basis by teachers and students to improve their learning experience. This chapter describes how the general problem of facilitating personalised learning concerns FeedbackFruits. It is divided into two sections. The first section discusses the needs of the client with regards to their existing platform and the problem they are facing. The second section explains why they would like to solve this problem.

% Steffan
\section{Needs}

The FeedbackFruits learning platform is structured around learning activities to facilitate a personalised learning experience. There are many types of activities such as documents, videos, presentations and questions. FeedbackFruits wants to know what types of activities their users would like to see implemented. Conversely, their users often have ideas about new types of activities or improvements upon existing ones. There needs to be a place where these ideas can be expressed, evaluated and improved.

Due to the growing amount of FeedbackFruits users, more developers are needed to keep up with the demand for new features. With the rise of open source software and open education in recent years, the client wants to make it possible for developers outside of the company to contribute to the FeedbackFruits platform to help meet the increased demand for features.

% Steffan
\section{Motivation}

In addition to collecting feedback from their users, FeedbackFruits would like to boost community engagement in improving education. It is the goal of the company to improve education for as many people as possible. The motivation behind stimulating community engagement is to create community-driven ecosystem for innovation in education.