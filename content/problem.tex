\chapter{Problem}

This chapter gives a description of the problem that was solved over the course of this project. It is divided into three sections. The first section describes the general problem that was addressed during this project. The second section explains how the problem applies the context of the client. The third section analyses how this problem can be decomposed into smaller problems.

% In the research report{ref:research-report:v1} a problem description was identified: ``How can the existing platform be extended in such a way that everyone in the FeedbackFruits community can contribute to the software development process?''.

% Steffan
\section{General problem description}

The client is a company that offers an online learning solution to help innovate education. They would like to gain insight into the wishes of their users and facilitate rapid development of new features by developers outside of the company. The online learning platform they offer is structured in a way that allows for easy integration of external services and new types of learning activities. The main philosophy behind the wishes of the client is that their users have a lot of valuable feedback, but it often gets lost in translation from feedback to feature or it might never reach the client at all due to the lack of means of communication.

% Wotuer!!!!!!!!!!!!! HODOR HODOR!
\section{Contextual problem description}
The client is motivated by community-driven innovation of education. For the client, this presents the problem of facilitating community engagement as well as innovation by development of new learning activities. These aspects must be united in order to solve the contextual problem. Innovation in the context of the client is achieved by the development of new learning activities. To engage the community, both the developers and the users of the clients platform have to be able to contribute to the innovation.


Users of the clients platform are mostly non-technical people. The research report identifies GitHub as the most important tool for developers in the collaborative software development process. This tool focuses on developers and is not designed to be used by non-technical users. While the users would like to improve the clients platform, it appears to be difficult to translate their ideas to the developers because of a knowledge gap. The research conducted during this project shows that there are no existing implementations of a solution to this problem. The solution should bridge this gap in knowledge. Therefore, the contextual problem description can be defined as: ``How can the existing platform be extended in such a way that everyone in the FeedbackFruits community can contribute to the software development process?''

%Wouter
\section{Problem analysis}
The most important aspects to this problem were identified in the research report as being: hype, requirements engineering, code and process updates. Hype is necessary to gain supporters that are willing to work on processing the feedback or feature requested by a user. To better understand the problem, the feedback should be decomposed into small challenges via requirements engineering. These challenges should be converted to working code, that through process updates inform the contributors of the current status. This provides a feedback loop that ensures the challenges are tackled correctly.



