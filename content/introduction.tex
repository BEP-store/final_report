% Steffan
\chapter{Introduction}

Albert Einstein once said: "I never teach my pupils, I only attempt to provide the conditions in which they can learn." This presents the idea of letting pupils do the learning themselves, as opposed to the traditional idea of a teacher imparting knowledge onto others. This idea has become increasingly popular in recent years with the rise of new methods in education such as 'flip the classroom' and 'blended learning'. Education as an industry seems to be moving towards personalised learning. To offer a personalised learning experience, a set of tools of required that can be adapted to facilitate such an approach. However many tools exist to facilitate the needs of the many, it remains difficult to gain insight into what tools are still missing.

This report describes the process of finding a method for gaining this insight and using it to improve the current state of education. The report is divided into 6 chapters. The first chapter explains the motivation behind the project in the form of a general problem. The second chapter introduces the client and how the general problem concerns them. The third chapter defines the problem in the context of the client. The fourth chapter describes the process used to find a solution to this problem and implement it. The fifth chapter presents the solution and compares it to the implementation. Finally, the sixth chapter presents the results of this comparison and recommendations to improve upon the implementation of the solution.

It is recommended to read the research report (see appendix \ref{app:research_report}) before reading chapters three and onwards. The research report contains valuable background information and is referred to throughout the final report.